\documentclass[12pt]{article}

\usepackage{notestyle}

\graphicspath{{./img/}}


\title{Note Web Application}
\author{Brendon Mendicino}



\begin{document}

\maketitle
\newpage
\tableofcontents
\newpage



\section{Javascript Introduction}
Javascript is backward compapatible, to be able to use the previous features is use the directive:
\begin{lstlisting}[language=javascript]
"use strict";
\end{lstlisting}
JS has primitive types and non-primitive types, JS is also and strongly typed language, the primitive types are: string, number, boolean, null, undefined. The non-primitive are the objects, which can be: array, function, user-defined.

The all possible false values in JS: \texttt{0, -0, NaN, undefined, null,''}, in JS there are two main comparison operators:
\begin{lstlisting}[language=javascript]
a == b    // equal, convert types and compare
a === b   // strict equal, inhibits automatic type conversion
\end{lstlisting}
In JS you can create variable with:
\begin{lstlisting}[language=]
// modern
let a = 10;    // can be changed
const b = 'a'; // constant

// old
var k = 9;
j = 30;
\end{lstlisting}
The difference between null and undefined, is that variable with null they old a value which is null, on the other way if a variable is declared and nothing is associated with it the value olds by default undefined.

A scope is defined by a \textbf{block}, which is created with \texttt{{ ... }}

There two kinds of \texttt{foreach} in JS, using \texttt{in} allows iterating over objects, while \texttt{of} allows iterating over iterable objects:
\begin{lstlisting}[language=javascript]
for (let a in object) {
  ...
}

for (let b of iterable) {
  ...
}
\end{lstlisting}
Using arrays:
\begin{lstlisting}[language=javascript]
let a = [1, 2, 'ok', false];
let b = Array.of(1, 2, true);
a.push(5);      // append an element
b.unshift(2);   // insert at the beginning

let copy = Array.from(a);  // shallow copy, it does not deep copy
\end{lstlisting}
The \textbf{destructuring assignment} can be done, it extracts the values from the mast left-hand side:
\begin{lstlisting}[language=javascript]
let [x, y] = [1, 2];
[x, y] = [y, x]     // swap
\end{lstlisting}
The \textbf{spread operator} (\texttt{...}) expands on iterable object into it's values:
\begin{lstlisting}[language=javascript]
let [x, ...y] = [1, 2, 3, 4];      // y == [2, 3, 4]

const a = [1, 2];
const b = [0, ...a, 3]; // [0, 1, 2, 3]
\end{lstlisting}
Spreding can be from the left or from the right, usually the spread operator is used for copying array:
\begin{lstlisting}[language=javascript]
const a = [1, 2];
const b = [...a];
\end{lstlisting}
A \textbf{string} is JS is an immutable type (like python) encoded in Unicode. The \textbf{template literals} can be done with the \textbf{tick} operator \texttt{``} (expression like Kotlin):
\begin{lstlisting}[language=javascript]
let name = 'Bre';
let sur = 'Mend';
// Template literal
let fullName = `${name} ${sur}`;
\end{lstlisting}











\end{document}


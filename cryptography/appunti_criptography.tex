\documentclass[12pt]{article}

\usepackage{notestyle}

\graphicspath{{./img/}}


\title{Notes Cryptography}
\author{Brendon Mendicino}



\begin{document}

\maketitle
\newpage
\tableofcontents
\newpage



\section{Introduction}
Ho voglia di piangere

\begin{definition}{Modulo Operator}{modulo-operator}
  \begin{align*}
    & a = b \pmod{n} : b = n \cdot q + a \\
    & a, b, q, n \in \mathbb{Z}
  \end{align*}
\end{definition}
\begin{definition}{Congurnce Modulo n}{congurnce-modulo-n}
  \begin{align*}
    & a \equiv b \pmod{n} \\
    & a \pmod{n} = b \pmod{n}
  \end{align*}
\end{definition}


\begin{example}{}{}
  Show that $ a \equiv b \pmod{n}$ if and only if $n$ divides $a - b$.
  \begin{proof}
    \[ r  = a \pmod{n} \implies a = aq + r, \quad q \in \mathbb{Z} \] 
    \[ a \pmod{n}  = b \pmod{n} = x \]
    \[ b  = nq_1 + x, a = nq_2 + x \]
    \[ \frac{b-a}{n}  = \frac{nq_1 + x - (nq_2 + x)}{n} = q_1 - q_2 \]
    \[ \boxed{ q_1, q_2  \in \mathbb{Z} \implies \frac{b-a}{n} \in \mathbb{Z} } \]
  \end{proof}
\end{example}

\subsection{Symmetric Cryptography}
A symmetric cryptosystem $\Pi$ consist of three algorithms:
\begin{itemize}
  \item Decryption
  \item Encryption
  \item Generation
\end{itemize}
\begin{definition}{IND-secure}{ind-secure}
  A system $\Pi$ can be defined \textbf{IND-secure} if, given two plain-text as inputs $(P_0, P_1)$ to $\Pi$ and by randomly choosing one of them, there is no better chance of $0.5$ to determine whether the ciphertext was generated from $P_0$ or $P_1$.
\end{definition}




\section{Randomness}
In mathematics there is no definition for randomness, in a paper published by Lehmer (1951) tries to give an idea of what random numbers can be: \emph{"A pseudo-random is a vague notion embodying the idea of a sequence in which each term is unpredictable and whose digits pass a certain test"}. One the methods involving random number is the \textbf{Monte Carlo} used, for example, in the computation of differential equation or the computation of $\pi$.

By Lehmar one way of computing a sequence is:
\[ u_n = f(u_{n-1}, u_{n-2}, \dots, u_{n-k}); \]
By using recursion it's possible to get the next number.
\begin{definition}{Lehmar Generator}{lehmar-generator}
  \begin{align*}
    & s_0 = \text{seed} \\
    & s_{i+1} \equiv a \cdot s_i + b \pmod{m}, i \in \mathbb{N}
  \end{align*}
\end{definition}
\hfill
\begin{example}{rand() in ANSI C}{rand-in-ansi-c}
  \begin{align*}
    & s_0 = 12345 \\
    & s_{i+1} \equiv 1103515245 \cdot s_i + 12345 \pmod{2^{31}}, i \in \mathbb{N}
  \end{align*}
\end{example}
\hfill
\begin{definition}{Pseudo Random Generator Function}{pseudo-random-generator-function}
  Where: $\mathbb{Z}_2 = {0, 1}$, which is a binary number, $\mathbb{Z}_2^{m} = [{0,1},\dots,{0,1}]$ is sequence of bits.

  PRNG is defined as:
  \[ G: \mathbb{Z}_2^{n} \rightarrow  \mathbb{Z}_2^{l(n)}, l(n) > n \text{(expansion factor)} \]
  Such that no adversary can succeed with probability $> 1/2$ if: given a sequence of random numbers that the cryptographer has, by flipping a coin he sends to the adversary:
  \begin{itemize}
    \item 0: sends a sequence $\mathbb{Z}_2^{l(n)}$ of random bits;
    \item 1: sends a sequence $\mathbb{Z}_2^{l(n)}$ of bits created with $G$, staring from $\mathbb{Z}_2^{n}$ random bits;
  \end{itemize}
  If the adversary can guess with probability $> 1/2$ if $G$ function is used then $G$ is \textbf{IND-secure}.
\end{definition}




\subsection{Stream Cipher}
There are two operation modes with stream cipher:
\begin{itemize}
  \item \textbf{synchronized mode}:
  \item \textbf{unsynchronized mode}:
\end{itemize}

\begin{definition}{Concatenation}{concatenation}
  \begin{align*}
    Enc_k(m_j) &= \text{bits of }IV_j || G(k, IV_j, 1^{|m_j|}) \\
    & <IV_j, G(k, IV_j, 1^{|m_j|}) \bitsetXor m_j>
  \end{align*}
  $1^{|m_j|}$ = unary notation, it means passing the length of the sequence of bits (it interpreted in base 1, which means that there are as much 1s as the length of the bits required).
\end{definition}


\subsection{Linear Feedback Shift Register}
In order to construct the bit stream, let's say there are $m$ registers which contain a single bit, at every clock the next bit of the key stream is extracted from the last register, also at every clock every register is shifted to the left. When shifting there is feedback loop that, when shifting, computes the next value of the initial register. One way of accomplishing this is using the \textbf{Fibonacci LFSR}.

The output is computed by:
\[ s_m = f(\vv{s}) = \sum_{j=0}^{m-1} s_j \cdot p_j \]
The coefficients are called \textbf{taps}, they resemble the way a tap opens or closes.

It's possible to represent the state of the flip-flops at every clock cycle using a matrix:
\[ \vv{L} = \left[ \begin{matrix}
  p_{m-1} & 1 & 0 & \dots & 0 \\
  p_{m-2} & 0 & 1 & \dots & 0 \\
  \vdots & \vdots & \vdots & \ddots & \vdots \\
  p_1 & 0 & 0 & \dots & 1 \\
  p_0 & 0 & 0 & \dots & 0
\end{matrix} \right]
\]
The registers have all possible state of $2^{m}$, after some time ($2^{m}-1$) that the $\vv{L}$ matrix is applied, there will come a state such that the sequence of the bits start to loop, the reason is that the set is finate.

\hfill

Let's image to use the LFSR to generate a cryptographic algorithm: we have a message n-bit long, and the ciphertext is the xor with the sequence produced by the LFSR, the key is shared and contains the initial state of the flip-flop the caracteristic polynomial. But this can be easily broken: the attacker uses KPA (assuming that he knows a segment of the cleartext), from this information he can recover the key. Let's assume $m=3$, and the key is $k=(\vv{p})$:
\[
  \begin{system} 
    s_3 = s_2 p_2 + s_1 p_1 + s_0 p_0 \\
    s_4 = s_3 p_2 + s_2 p_1 + s_1 p_0 \\
    s_5 = s_4 p_2 + s_3 p_1 + s_2 p_0
  \end{system} 
\]
Where $\vv{p}$ is the unknown variable, this linear equation can be easily solved, to get $s_i$ the attacker knows some part of the cleartext, and he can extract them by $c_i \oplus x_i$, where $x_i$ is known part of the cleartext and the attacker knows \emph{at least} $2m$ consecutive bits.

LFSR should not be used in cryptography but in the past people used a combinations of this generators, called \textbf{Geffe Generator}, but also with this kind of generator a KPA is possible.


\subsubsection{Galois LSFR}
The difference with the Fibonacci one is that in this case there are polynomial multiplications. The state of the flip-flop represent a polynomial. The state is:
\[ s(x) = g_0 + g_1x + \dots + g_{m-1}x^{m-1} \]
The reason why polynomials are used is that Galois theory is used in many algorithms.



\subsection{Permutation}
A permutation is defined as function such that:
\[ f: \mathbb{Z}_2^n \rightarrow \mathbb{Z}_2^n \]
...













\newpage
\section{Openssl}
Openssl has two versions 1.x and 3.x and 1.x will be dropped soon, but many applications still use it. The low level software implementations of the algorithm was a big mess, so a layer was created on top of it called \emph{EVP Crypto API}, that just takes in the parameters and does a translation handling all the data.



Typical use of openssl:
\begin{enumerate}
  \item include libraries
  \item load facilities: load the functions required
  \item create the context: select the tools, like a certain symmetric algorithm
  \item initialize the context: assign IV, nonce, key...
  \item operate on the context: provide the data on which the machine will work
  \item finalize on the context: perform the concluding operations on the last output, like putting the padding, or the length of the digest
  \item free the context: all the objects are \emph{one time objects}, at the end of the operations the objects need to be freed;
  \item free facilities
\end{enumerate}
Usually the mode of use of the libraries is the incremental mode, which allow get small blocks a data encrypted.

To get an object the library is called which will return the function pointer to the implementation.
\begin{lstlisting}[language=c]
EVP_CHIPER *c = EVP_bf_cbc();
\end{lstlisting}


\subsection{Symmetric Encryption}
The \textbf{context} where the data and algorithms are stored are the \texttt{EVP_CIPHER_CONTEXT} data structure. 
Example of a string saved in memory:


\subsection{Big Number}
In OpenSSL to handle big number (biggere that the CPU registers), the \texttt{bn.h} library is used to handle this kind of numbers, for example to represent the keys of the RSA algorithm \texttt{BIGNUM}s are used.

A way to generete a big number is by using:
\begin{lstlisting}[language=c]
BIGNUM *a = BN_new();
BIGNUM *b;

BN_copy(a, b);
BIGNUM *c = BN_dup(b);

BN_free(a);
// ...
\end{lstlisting}
The bignum needs to always be freed.

The library offers a way to convert the big numbers into other types, like into a string of raw bytes (\texttt{BN_bn2bin()}), or from a decimal string to a big number (\texttt{BN_dec2bn()}), ...


\subsection{Asymmetric Cryptoraphy}
In OpenSSL 3.0, the EVP interface allows to generate the keys and use them in the algorithm, but in OpenSSL 1.0, the functions use are \verb|RSA_generate_key_ex| (the version without the \texttt{ex} is deprecated), \texttt{RSA\_public\_ecrypt}. On the other hand with EVP we can use the basic way using of using the asymmetric encryption which is not very modifiable, or it's possible to use the \texttt{context} or the \texttt{OSSL\_PARAM} data structure, which allows to very detailed way of secifing the parameters.

It's also possible to sign some data, there are steps to be taken in order to sign:
\begin{enumerate}
  \item create context;
  \item \texttt{EVP\_DigestSignInit}: which takesas input the private key and the digest alogrithm;
  \item \texttt{EVP\_DigestSignUpdate}
  \item \texttt{EVP\_DigestSingFinal}
\end{enumerate}

How to generate a \emph{RSA private key}:
\begin{lstlisting}[language=c]
RSA *rsa_keypair = RSA_new();
BIGNUM *bne = BN_new();
BN_set_word(bne, RSA_F4);

RSA_generate_key_ex(rsa_keypair, 2048, bne, NULL);
\end{lstlisting}





















\end{document}

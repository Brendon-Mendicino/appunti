\documentclass[12pt]{article}

\usepackage{notestyle}

\graphicspath{{./img/}}


\title{Notes Cryptography}
\author{Brendon Mendicino}



\begin{document}

\maketitle
\newpage
\tableofcontents
\newpage



\section{Introduction}
Ho voglia di piangere





\section{Openssl}
Openssl has two versions 1.x and 3.x and 1.x will be dropped soon, but many applications still use it. The low level software implementations of the algorithm was a big mess, so a layer was created on top of it called \emph{EVP Crypto API}, that just takes in the parameters and does a translation handling all the data.


Typical use of openssl:
\begin{enumerate}
  \item include libraries
  \item load facilities: load the functions required
  \item create the context: select the tools, like a certain symmetric algorithm
  \item initialize the context: assign IV, nonce, key...
  \item operate on the context: provide the data on which the machine will work
  \item finalize on the context: perform the concluding operations on the last output, like putting the padding, or the length of the digest
  \item free the context: all the objects are \emph{one time objects}, at the end of the operations the objects need to be freed;
  \item free facilities
\end{enumerate}
Usually the mode of use of the libraries is the incremental mode, which allow get small blocks a data encrypted.

To get an object the library is called which will return the function pointer to the implementation.
\begin{lstlisting}[language=c]
EVP_CHIPER *c = EVP_bf_cbc();
\end{lstlisting}





















\end{document}

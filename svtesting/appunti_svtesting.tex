\documentclass[12pt]{article}

\usepackage{notestyle}

\graphicspath{{./img/}}


\title{Notes Security Verification and Testing}
\author{Brendon Mendicino}


\begin{document}

\maketitle
\newpage
\tableofcontents
\newpage

\section{Introduction}
... 

\section{Vulnerability Assessment}
The NIST proposed the risk management framework, which allows any company that doesn't have a cybersecurity expert to follow some steps to finalize the vulnerability assessment.

FRAME 
\textbf{begenning:} we need to decide our assets and our infrastructure with an estimation, then there is the \textbf{risk assessment} where we idenitify the threats, weaknesses, vulnerability, consequences and the output is all the possible risks.

Respond
We estimate how effective out strategies are

Monitoring
Always check for risk that where not considered

NIST steps for vulnerability assessment:
threat-oriented: start from the attackers, from that we find the sources and the king of attacks performable, we try to find 

asset/impact-oriented: vulnerability scanning and build our information modles, ...

When doing vulnerability assessment:
1. we have to plan and define a scope:
- Define the objective and the purpose
- Black: we only see what an attacker can see
- White: prespective of the defender
- Grey: we know some parts of the attack but not too much (usually when we rely on a third party company)
- Scope: it is important the define the scope of what we want to assess, this done to reduce our costs and time
- Frequency: it is important to define the frequency of the analisys

2.
Gather as much information as possible:
services running, open ports, OS, ... also knwoing where security monitoring are like: firewall, IPS, ...

3.
scanning, detection and assessment,
when we talk about scanning, we are talking about \textbf{automation}, many tools are used and run one after the other.

4. 
Reporting:
The company pays to have a report at the of the assessment, where some advice are given to solve the problems. A company needs some \textbf{penetration testing} after the automation part.


\subsection{Lab}
\subsection{nmap}
The main tecqnique to scan a host are:
- ping scan: one of the first things the firewall will filter
- ARP scan: access to the network is required
- TCP SYN/ACK: the port are alive and no one has performed \textbf{hardening}

Port scanning:
the answer will be: open, close, filtered, or mixuture of those

Scanning approches
SYN scan: user sends SYN, server answer SYN/ACK, when permorning a scan the client won't send back the ACK, we are only interested in checking if a server is alive

TCP connect: complete the 3-way handshake, but the some logs are maintained and it is slower

TCP flags scanning: we may use the other flags of TCP:
- Null scan
- FIN scan 
- ...

UDP scan:
 ...


Detect the version of the OS, this is done by looking at the response behavior

TCP Sequence predictability ...,

nmap also has a Lua scripting language to perform more sophisticated information, add test, additional features.

\subsection{GVM}












\end{document}

%% vim: ts=2 sts=2 sw=2 et

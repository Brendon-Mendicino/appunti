\documentclass[12pt]{article}

\usepackage{notestyle}

\graphicspath{{./img/}}


\title{Notes Optimization}
\author{Brendon Mendicino}


\begin{document}

\maketitle
\newpage
\tableofcontents
\newpage

\section{Linear Programming Model}
It is a way to represent a problem as linear model, to create this model we need:
\begin{itemize}
  \item variables: used to describe the problem
  \item objective function: find the maximum or minimun of given function
  \item constraints: limit the variable in a range
\end{itemize}
In order to create a linear model objective functions and constraints must be \textbf{linear}. If we can describe the problem as linear system, there is a method to find the optimal solution, which is called \textbf{simplex methdo}.
\begin{align*}
  & \text{real life problem} \longrightarrow_{\text{modeling}} \\
  & \text{LP Problem} \longrightarrow_{\text{simplex}} \\
  & \text{linear solution} \longrightarrow_{\text{decoding}} \\
  & \text{real life solution}
\end{align*} 

\newpage
\section{Modeling}
The steps of modeling are:
\begin{itemize}
  \item idenitfy variables
  \item write objective function
  \item write constraints
\end{itemize}
The variables are a set of entities that describes the solution of the problem, given those:
\begin{itemize}
  \item it is immediate the get the objective function 
  \item it is immediate to check if the solution respects the constraints
\end{itemize}

\begin{example}{The Knapsak Problem}{}
  \textit{A bunch of friends is organizing an excursion and decies to put everithing in a single knapsack with capacity 10Kg.}
  \\ \\
  The knapsack may be filled with:
  \begin{itemize}
    \item chocolate (500g)
    \item fruit juice (1l)
    \item beer (0.33l)
    \item snadwiches (100g)
    \item mineral water (1l)
    \item cookies (500g)
  \end{itemize}
  Each product is given a score:
  \begin{itemize}
    \item chocolate (10)
    \item fruit juice (10)
    \item beer (6)
    \item snadwiches (3)
    \item mineral water (20)
    \item cookies (8)
  \end{itemize}
  It was decided to garantee a minimun amount of each product:
  \begin{itemize}
    \item chocolate (2)
    \item fruit juice (2)
    \item beer (6)
    \item snadwiches (10)
    \item mineral water (1)
    \item cookies (2)
  \end{itemize}
\end{example}


Given the full model, we insert it into the simplex and it returns back the best optimal solution

Example:
The steel plant

Example:
From chicken farm to fast food.

The varaibles are the quantity of chicken going from every farm to each single shop. We can represent this a matrix with columns the farms and as for rows the quantities of chicken


Minimize and absolute value, we want to minimize the difference between the target and the value
\begin{align*}
  & \min{z} = |v - t| \\
  & \min{z} = |a| \\
  & |a| = \max{\{a, -a\}} \\
  & \min{z} = \max{\{a, -a\}} 
\end{align*}
we introduce an auxiliary variable $Y$
\begin{align*}
  & \min{z} = \min{Y} = \max{\{X_1, X_2\}} \\
  & Y >= X_1 \\
  & Y >= X_2 \\
  \\
  & \min{Y} \\
  & Y >= a \\
  & Y >= -a
\end{align*}

Example: 
...
We can express binary variables, $if X_1 > 0 then X_2 = 0$ and $if X_2 > 0 then X_1 = 0$. We can express them with binary variable $Y_1 = 0, 1$ $Y_2 = 0, 1$

BIG M, where M is constant, big fixed number. We want to express
\begin{align*}
  & Y_1 = 0, 1 \\
  & Y_2 = 0, 1 \\ 
  & Y_1 + Y_2 <= 1 \\ 
  & X_1 <= M_1 Y_1 \\ 
  & X_2 <= M_2 Y_2 
\end{align*}
If $Y_1$ is 0 $X_1$ is bount to 0, if $Y_1 = 1$ $X_1$ is virtually unbounded, the same for $X_2$ and $Y_2$. $M_1$ and $M_2$ are large constants. During a software simulation it reasonable to choose the varaibles in a range that won't be reachable.


\subsection{Fixed Values}
Problems where we have only a fixed amount of resources. We manage them by using several logical variable for every variable that is fixed.
\begin{align*}
  Y_1 + \dots + Y_k = 1 \\
  X = p_1 Y_1 + \dots + p_k Y_k
\end{align*}
The first equation imposes that only of the $Y$ can be one. The second equation imposes that $X_j$ can only have one of the quantities $p1, p2, \dots, p_k$.


Example 6:


A1, ..., A5
B1, ..., B5
C1, ..., C5
D1, ..., D5
we remove the years we cannot invest in.

we derive the first constraints starting from the first year. Wich is the sum of investments less than the total money.





\end{document}

%% vim: ts=2 sts=2 sw=2 et
